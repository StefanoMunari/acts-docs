\section{City Instances}

In order to use valid instances for our simulations, we automatized the build
of configuration files for a given city with a Python program (Slartibartfast)
and a collection of Bash scripts.

Slartibartfast does its work in two steps: in the first it creates
descriptive information about the city (e.g., number of streets, their
disposition, number of pedestrians, ...) and then it creates JSON configuration
files that are valid with respect to the input generated during the first step.

These cities are all structured as a grid, therefore:

\begin{itemize}
    \item in each corner two streets meet;
    \item at the edges, three streets meet (the two on the edge and other one
        towards the inside of the grid);
    \item otherwise, intersections have four exits.
\end{itemize}

Another simplification we adopt is to always have a street composed in the
following way:

% TODO: Replace with figure?

\begin{verbatim}
F0 F1 B0 B1 R0 R1 B2 B3 F2 F3
\end{verbatim}

Where \texttt{FX} are sidewalk lanes, \texttt{BX} are bikeway path lanes and
\texttt{RX} are roadway lanes. Furthermore, two lanes can be treaded in the
same direction iff their suffixes have the same parity (e.g., B0 and B2
could be both directed from North to South, whereas B0 and B3 have opposite
orientation).

Slartibartfast produces JSON with possible redundant information: this has
been handy to deal with the different needs of different parts of our system.
Clearly, some information can be completely ignored by some architectural
macro-components: for example, the backend application layer is completely
unaware of the disposition of the ways in a street.

Finally, a collection of scripts copies all the information to our source code
repository, so it becomes available to us for running simulations.
