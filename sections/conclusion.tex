\section{Conclusion}

ACTS has been a project which made us reason thouroughly about different
aspects of a concurrent and distributed system.
In particular, we had the chance to experiment and open our minds with
technologies which are mostly used by their niches, like Ada and Elixir.

When we first started ACTS, we could have decided to head towards different
goals. For instance, we could have decided to use one single technology for the
whole project, but then we wouldn't have realized how it is to be with high
heterogeneous systems, which is an actual issue in the industry.

\subsection{Technologies}

As for the technology choice, we intend to provide a series of opinions ensued
from our experience:

\noindent\textbf{Ada} allows to write solid and reliable systems, but it is
hard to develop a general-purpose application with it, for several reasons. Its
learning curve is definitely steep and it is not easy to master Ada in order
to apply common OOP design patterns or even to perform string processing or
operations which are quite straightforward in other programming languages.
Also, Ada lacks of a strong community support like other technologies do, and
therefore sometimes it has been difficult to find a solution for easy
problems. However, we have to recognize that Ada's strong typing and its
concurrency model helped us in many occasions;

\noindent\textbf{Elixir} turned out to be actually both productive, fast and
reliable as claimed by its website. In fact, many of its features are very
useful when reasoning about distribution, and we were able to develop both the
middleware and the broker subsystems in a relatively short amounts of time.

\noindent\textbf{RabbitMQ} and \textbf{Redis} were straightforward to use and
configure.

\noindent\textbf{Docker} certainly was handy for our needs. However, we had to
automate ourselves a substantial part of the configuration procedures, because
we wanted to achieve flexibility and different modes (development, production
and test) for ACTS.

\noindent\textbf{AngularJS 4} eased sensibly the development of our frontend.
However, during the development of our system AngularJS 2 was effectively set
aside by Google and we had to port our frontend to AngularJS 4 to continue
enjoying a live and active community and to avoid using deprecated libraries.

\subsection{Expectations and Actual Outcome}

We were able to reach the goals we set ourselves at the beginning of this
project, despite it has been a cookie tougher than expected.
In fact, our simulator works as expected and we had the chance to experiment
and see with our own eyes how different technologies interact when brought
together to compose up a general-purpose application.
We also had the chance to combine more than one academic project in order to
develop something more meaningful: in fact, we integrated an Artificial
Intelligence project in our system so that our travelers were able to find a
route to reach their destination by means of different heuristics.
