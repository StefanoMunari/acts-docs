\section{Problem Analysis}

This section will expose several problems we took in account during the
initial phase of the project:

\begin{itemize}
\item \textbf{Domain Analysis}: description of the applicative context we
  considered to model
\item \textbf{Distribution}: a need to distribute the system arises from the
  domain analysis. In this section we will discuss issues that we encountered
  while thinking how to distribute the city traffic simulator
\item \textbf{Concurrency}: it is reasonable to believe that some entities
  will act concurrently in the system, since the city traffic intrinsically
  contains more than one control unit
\item \textbf{Time}: we finally outline which time issues we considered for
  this project
\end{itemize}

\subsection{Domain Analysis}
The goal of this project is to represent the traffic of a city. Among all the
different options we could choose, we decided to simulate a lively city of
the United States of America, like San Francisco.

Therefore, we can make some assumptions about this reality:

\begin{itemize}
\item the city is likely to have a lot of commuters;
\item the city traffic can grow in certain days due to special events or
  recurrences;
\item cars are used mostly by a single person;
\item people mostly commute between two city spots, where we assume there
  will be a facility (a house, an office or something else) for each one;
\item the city-planning is Roman-like, i.e. the streets have orthogonal
  intersections (\href{https://www.google.it/maps/place/San+Francisco,+California,+Stati+Uniti/@37.7766566,-122.4330836,16z/data=!4m2!3m1!1s0x80859a6d00690021:0x4a501367f076adff}{example}).
\end{itemize}

Even if our reference city\footnote{with the term \emph{``reference city"}
we're talking about but a city which has similar characteristics to San
Francisco} has many means of transport, we will consider only a few meaningful
ones (besides \textbf{pedestrian} traffic): \textbf{bikes}, \textbf{cars} and
\textbf{buses}.

Every street stretch could have sidewalks, bicycle paths and motor vehicles
lanes. In our reference city road users must respect the following traffic
code:

\begin{itemize}
\item motor vehicles must stop in order to let pedestrians and bikes walk over
  respective zebra crossings;
\item each road user has to proceed in his/her street part (pedestrians can't
  tread a road in the middle of a car lane);
\item each crossroads can either be or not be regulated by semaphores;
\item road users must not proceed faster than an eventual speed limit posted
  on a street stretch;
\item vehicles can't enter a one-way in the wrong direction;
\item vehicles must give the way to other ones if it is requested by
  regulatory sign (STOP or YIELD);
\item vehicles must not turn in directions that are eventually forbidden by
  road signs;
\item vehicles must turn in directions that are eventually specified by road
  signs.
\end{itemize}

\subsection{Distribution}
%TODO add Distribution section

\subsection{Concurrency}
%TODO add Concurrency section

\subsection{Time}
%TODO add Time section

