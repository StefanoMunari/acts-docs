\section{Problem Analysis}

This section will expose several problems we took in account during the
initial phase of the project:

\begin{itemize}
\item \textbf{Domain Analysis}: description of the applicative context we
  considered to model
\item \textbf{Distribution}: a need to distribute the system arises from the
  domain analysis. In this section we will discuss issues that we encountered
  while thinking how to distribute the city traffic simulator
\item \textbf{Concurrency}: it is reasonable to believe that some entities
  will act concurrently in the system, since the city traffic intrinsically
  contains more than one control unit
\item \textbf{Time}: we finally outline which time issues we considered for
  this project
\end{itemize}

\subsection{Domain Analysis}\label{sec:pa-domain}
The goal of this project is to represent the traffic of a city. Among all the
different options we could choose, we decided to simulate a lively city of
the United States of America, like San Francisco.

Therefore, we can make some assumptions about this reality:

\begin{itemize}
\item the city is likely to have a lot of commuters;
\item the city traffic can grow in certain days due to special events or
  recurrences;
\item cars are used mostly by a single person;
\item people mostly commute between two city spots, where we assume there
  will be a facility (a house, an office or something else) for each one;
\item the city-planning is Roman-like, i.e. the streets have orthogonal
  intersections (\href{https://www.google.it/maps/place/San+Francisco,+California,+Stati+Uniti/@37.7766566,-122.4330836,16z/data=!4m2!3m1!1s0x80859a6d00690021:0x4a501367f076adff}{example}).
\end{itemize}

Even if our reference city\footnote{with the term \emph{``reference city"}
we're talking about but a city which has similar characteristics to San
Francisco} has many means of transport, we will consider only a few meaningful
ones (besides \textbf{pedestrian} traffic): \textbf{bikes}, \textbf{cars} and
\textbf{buses}.

Every street stretch could have sidewalks, bicycle paths and motor vehicles
lanes. In our reference city road users must respect the following traffic
code:

\begin{itemize}
\item motor vehicles must stop in order to let pedestrians and bikes walk over
  respective zebra crossings;
\item each road user has to proceed in his/her street part (pedestrians can't
  tread a road in the middle of a car lane);
\item each crossroads can either be or not be regulated by semaphores;
\item road users must not proceed faster than an eventual speed limit posted
  on a street stretch;
\item vehicles can't enter a one-way in the wrong direction;
\item vehicles must give the way to other ones if it is requested by
  regulatory sign (STOP or YIELD);
\item vehicles must not turn in directions that are eventually forbidden by
  road signs;
\item vehicles must turn in directions that are eventually specified by road
  signs.
\end{itemize}

\subsection{Distribution}\label{sec:pa-distribution}
As we pointed out in section \ref{sec:pa-domain}, our system will be composed
of streets that are joined together by crossroads.

Hence, in this section we'll discuss about how we analyzed distribution issues.

\paragraph{Requirements} \mbox{} \\

In the following list we'll expresse qualities our system might have to
leverage distribution.

\begin{itemize}
\item The system should be distributed: we expect the simulator to manage many
  streets and much traffic. Thus, the system is probably going to balance load
  in a better way if it allots nodes to geographical partitions of the city
\item The system should appear as a single unit to its end user, in order to
  have an acceptable degree of transparency
\item The system might be composed of heterogeneous parts: whilst this issue
  isn't mandatory, it makes the system more realistic
\item The system should be scalable: considering the nature of our project (a
  city simulator), we would like it to be scalable in possible directions
  (e.g. number of road users, city size, etc.)
\item There should be a clearly defined protocol for the communication between
  each part of the system
\item When possible, the communication between two nodes of our system will be
  asynchronous: this quality leverages scalability
\item The system should boot/terminate consistently and ordered: with a fixed
  algorithm it is easier to ensure that all nodes are ready to execute/stop the
  system
\item The system should be able to save a consistent global state
\item The system should be fault tolerant, i.e. \emph{reactive} to errors
\item The system should be able to replicate and/or restore parts of itself,
  in order to be more fault tolerant
\item The system should be able to locate each part of itself through a system
  of unique names
\item The end user should not be aware of part replication/restoration
\end{itemize}

\paragraph{Problems} \mbox{} \\

Several problems arose from this issues:

\begin{itemize}
\item How can we effectively boot the system in a neat and systematic manner?
\item How can we take a snapshot that is actually consistent in a distributed
  system?
\item How can we effectively shutdown the system in a neat and systematic
  manner?
\item The system might be seen in a control panel as if it were a film:
  \begin{itemize}
  \item what will it be able to show?
  \item will it offer different modalities an user can consume?
  \item how can we make the control panel see the traffic simulator as a
    single unit?
  \item will the stream show live events, deferred events or both?
  \end{itemize}
\item How can we set up an efficient and reliable naming resolution module?
\item How should we scatter the system across a set of nodes? How and which
  entities should we place in each node?
\item How errors will be managed by our system so that it can reach a certain
  degree of fault tolerancy?
\item Are the answers at the previous questions providing a scalable solution?
\end{itemize}

\subsection{Concurrency}
Looking at the applicative domain, there are several situations that clearly
express concurrent interactions between entities, for example more road users
that proceed in the same street stretch.

Like we did for distribution (see section \ref{sec:pa-distribution}), we'll
show which concurrency issues we took in account during the initial phase of
the project.

\paragraph{Requirements} \mbox{} \\

In the following list we'll expresse qualities our system might have to
leverage concurrency.

\begin{itemize}
\item The system will present concurrency between some of its entities
\item The critical sections of each entity should be identified and accesses to
  them will have to be studied carefully
\item Time spent in critical sections will be reduced as much as possible
\item Blocking operations that are placed inside critical sections will be
  considered errors
\item We have to be able to show evidence of the fact that our system avoids
  deadlocks and livelocks.
\item Entities that act like servers should not also act like clients (and
  viceversa)
\item Causal dependencies between entities have to be found out
\item Each valid state of an entity should be studied carefully so that
  protocols between entities operate in a correct way
\item Non-valid entity states should be pointed out so that we are able to
  think possible countermeasures to recover a preceeding valid state
\item Since in section \ref{sec:pa-distribution} we express our willingness to
  be capable of obtain consistent snapshot, each entity could have a concurrent
  part that handles the incoming and outgoing flow of information
\end{itemize}

\paragraph{Problems} \mbox{} \\

Concurrency requirements led us to make some questions about concurrency.

\begin{itemize}
\item How can we recognize that a situation is suitable to present concurrency?
\item How concurrent accesses will be managed? Will it make sense to discern
  side-effect accesses (\emph{procedures}) from reading accesses
  (\emph{functions})?
\item Which entities has to be active?
\item Which entities has to be reactive?
\item Which entities has to be passive?
\item Should we use a formal technique to give proof of deadlock/livelock
  avoidance?
\item Should we use a formal technique to give proof of some properties of the
  system that depend on concurrency?
\item Do we have the possibility to incur a non-valid state? If so, how do we
  handle the problem?
\end{itemize}

\subsection{Time}
Even if a traffic simulator would be appropriate for a real-time application,
we'll limit us to think about consistency in a general way though putting our
best effort to make choices that minimize the drift among nodes and entities.

Thus, we'll strive to solve in the most realistic way the following issues:

\begin{itemize}
\item How can we simulate road users proceeding (regarding to travelling
  speed)?
\item How can we simulate crossroads (regarding to road users' arrivals time)?
\item How long do people stay in facilities for? How much precise do these
  intervals have to be?
\item How are semaphores going to be able to synchronize? Which of them decide
  how much time do they have to wait for?
\end{itemize}
