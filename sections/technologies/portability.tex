\subsection{Portability}
We have built a heterogeneous system
(Elixir, Ada, C++, Python, Redis, Postgres, RabbitMQ)
and made it completely portable by packaging each service in a Docker
container.
Also, using docker-compose we composed and linked the services together while with docker-swarm
we orchestrated the entire system. By using docker-swarm in conjunction with docker-compose we
have been able to deploy each city individually as a single stack of services. Furthermore, we
have defined specific restarting policies for each service to recover them in case of failures
(e.g., restart the service with a maximum number of tries and with a specific delay between each restart).
Docker-swarm is an interesting tool to orchestrate our system because it enables to define rescheduling policies
for docker nodes and scale factor for specific services.

Finally, we provided three different deployment environments for each of the
containers:
\begin{enumerate}
  \item \textit{Development}: allows the developer to
    log into the specific service as it were its local environment
    with all the dependency installed and the multi-container configured;
  \item \textit{Test}: automatically runs all the tests specified by
    the developer;
  \item \textit{Production}: deploys the environment which will be used by
    the final user. Note that the final user has only to issue one command
    which creates (if not yet created) and runs the multi-container system.
    Then the user can access its control panel through a web interface.
\end{enumerate}
