\subsection{Backend::Application}
We chose the Ada 2012 language \cite{taft2012ada}
to implement the application layer of the backend.

The Ada language standard was originally designed in the early 1980s to promote
sound software engineering. With the advent of Ada 2005 \cite{taft2006ada},
new features
remarkably improve the support for high-integrity systems in three major
areas: object-oriented programming, tasking, and real-time features.

The latest version (Ada 2012) brings further improvements, among which for-each
loops (\texttt{for .. of}) which enhances both code readability and
productivity.

We believe that the properties of the application layer
that we are looking for, especially for concurrency, could be supported by the
following features provided by Ada:
\begin{itemize}
  \item Protected Object (PO): a structured mechanism to handle
        mutual exclusive access to shared resources.
        It distinguishes functions (read-only
        operations) from procedures (read-and-write operations). It also
        implements the eggshell model, which allows to avoid race condition and
        polling in a concurrent environment by providing two level of access
        to its state;
  \item Avoidance of silent task termination;
  \item Possibility to define language profiles (using pragma restrictions);
  \item Timing events: to define async/time-based events with associated
        handlers;
  \item Typing: it is possible to define our own types, also restricting
  the standard types to check the range of values at static time;
  \item Buffer overflow and dangling reference prevention;
  \item Object orientation, packages and child units to organize code in a
  structured (hierarchical) way.
\end{itemize}

Other important properties of the language that will help us to avoid
error-prone code are:
\begin{itemize}
  \item \textbf{Safety}: writing programs with high assurance does not
        introduce hazards;
  \item \textbf{Reliability}: preventing errors with detection at compile time
        (if possible);
  \item \textbf{Static typing}: type checking at compile time.
\end{itemize}

\subsubsection{Build}
Unfortunately, we have not be able to use the GNAT Project Manager (GPR)
to build the core project. Our application uses an AI written in C++,
to integrate it into the build of the project we implement a custom program
which first generates the Ada spec of the AI and then
links the AI's object files to the Ada entrypoint through the g++ linker.

\subsubsection{Tests}
We used AUnit for the application layer unit tests.
Also, we widely took advantage of dependency injection in our code in order to
be able to mock dependencies when testing packages. Therefore, there will be
some constructors or factory methods with a long list of parameters, most of the
which can be classified as dependency injections.

Since we know
the reason behind these injected parameter,
we don't think this practice degraded too much our code quality.
However, we recognize that setting these
injections directly inside the tests (since they are child packages)
probably could lead to a more cleaner implementation.


The testing process has been completely automate through makefile, python and
GNAT Project Manager (GPR): a developer can specify a set of test which he wants
to run, the system takes cares of compiling and running the specified subset of
tests (e.g. a set of test suites).

During the development of the project we used an AWS EC2 instance
equipped with Jenkins and Docker to automate the testing process with
reports provided to the developers through email. The developers were notified
only in case of broken builds or failed tests.

Also, the build of this document has been automated by leveraging the
\href{https://hub.docker.com/}{DockerHub},
a cloud based registry which provides free automated builds.
It notifies a server after each build, the server formats the received message
and then sends it to a specific channel of the team on slack (our team chat).
