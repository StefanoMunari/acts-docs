\subsection{Application::Back End}
The Ada 2005 language\footnote{Benjamin M. Brosgol \textit{Ada 2005: A 
Language for High-Integrity Applications}, CROSSTALK - The Journal of Defense 
Software Engineering, Hill AFB, UT, USA, vol. 19, pp. 8-11, August 2006} 
has been chosen to implement the application back end.

The Ada language standard was originally designed in the early 1980s to promote
sound software engineering. The latest version (Ada 2005) offers particular 
innovations that improves support for high-integrity systems in three major 
areas: object-oriented programming, tasking, and real-time features.

We have reason to believe that the properties of the application layer 
that we are looking for, particularly for concurrency, could be supported 
by the following features that Ada provides:
\begin{itemize}
    \item Protected Object: a structured and efficient mechanism to handle 
shared resources access
    \item Locking and scheduling policies
    \item Avoidance of race conditions during system initialization
    \item Avoidance of silent task termination
    \item Possibility to define language profiles (using pragma restrictions)
    \item Timing events: to define async/time-based events with associated 
handlers
    \item Buffer overflow and dangling reference prevention
\end{itemize}

Other important properties of the language that will help to avoid error-prone 
code:
\begin{itemize}
    \item Safety: writing programs with high assurance does not introduce 
hazards 
    \item Reliability: preventing errors with detection at compile time 
(if possible)
    \item Static typing: type checking at compile time
\end{itemize}
% should we \textit{...} these three features like in Technologies::Middleware?
