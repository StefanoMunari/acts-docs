\subsection{PEAP}
Regarding the application server and the frontend, we used a stack of
technologies we found online called PEAP. The acronym stands for
\textit{Phoenix-Elixir-Angular-Postgres} which are respectively the
server-side web framework, its language, the client-side web framework and the
database technology.

In this section we will quickly give overview on Phoenix and Angular, without
spending further words on Elixir and avoiding to provide further details on
Postgres. In fact, the database technology was not a key factor in our project,
since we focused on other problems; also, it is quite straightforward to
replace Postgres with another database by using Elixir's Ecto (which is a
database DSL) abstractions.

\subsubsection{Phoenix}

Phoenix is a web framework which focuses its efforts on being productive,
reliable and fast:

\begin{itemize}
  \item \textit{Productive}: Phoenix is quite simple to use once one gets the
    hang of it. Web requests are piped along different functions which
    transform the initial request in a response which will be returned back to
    the users browser. Real-time features like Channels are easy to use and to
    integrate with other technologies;
  \item \textit{Reliable}: Phoenix runs on the BEAM VM and therefore can easily
    scale even whne high loaded;
  \item \textit{Fast}: Numerous benchmarks showed how Phoenix outperforms other
    web frameworks (and our experience in this project later confirmed this
    claim).
\end{itemize}

Even if we used several features of Phoenix, we will focus on describing one:
Channels. As said before, Channels are (soft) real-time Elixir processes which
implement a publish-subscribe mechanism.
Anything could be able to connect to a Channel by subscribing a Socket to one
or more topics; in our case, we connected the user's client to the application
server by means of the phoenix.js library and then making our clients subscribe
to different topics according to what they were interested in.

\subsubsection{Angular}

Angular is a web framework from Google. It is used for helping the development
of frontend views in cross-platforms environment by adopting a MVVM model.
We use Angular 4.0, which is the most recent version of the framework up to
now; actually, we started with 2.0, but then we later decided to switch to
4.0.

Angular 4.0, like the 2.0 version, employs Typescript as its programming
language. This feature was convenient for us, since it helped to harness the
notorious Javascript unsafety.
We also used a series of other Javascript libraries to render the view and
develop the user client, with svg.js among these: in fact, we draw our cities
with SVG components, making it scale well and providing nice features like
zoom.
