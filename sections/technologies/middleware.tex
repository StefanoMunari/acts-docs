\subsection{Backend::Middleware}
% \footnote{Erlang is a general-purpose, concurrent, garbage-collected programming language}
We chose Elixir 1.5 (an Erlang\footnote{Erlang is a general-purpose, concurrent, garbage-collected programming language} dialect) to implement
our middleware. Elixir comes with a series of features like metaprogramming,
advanced build tools and Erlang compatibility which allow Elixir to make the
Erlang ecosystem more accessible for newcomers.
However, the main reasons to choose Elixir/Erlang are the BEAM virtual machine
and the OTP library.
BEAM (Bogdan's Erlang Abstract Machine) is the Erlang runtime which provides an
enviroment to build portable, distributed, fault-tolerant and soft real-time
applications.
Since Erlang does not have a formal specification for its runtime we will
base our motivations on a comparison between BEAM and the JVM Hotspot
implementation \cite{poolcomparison}.

\paragraph{Notation}
When talking about middleware and Elixir, we will often use the term
\textbf{process} instead of lightweight thread, as well as \textbf{application}
in place of ``architectural macro-component''.
We will strive to adhere
to the Erlang/Elixir notation, which has these old fashioned/slightly inaccurate
names since it was born and developed for another context, i.e., telephony.

\paragraph{BEAM and OTP}
\begin{itemize}
  \item \textit{Concurrency model}: the Elixir concurrency model relies on
        asynchronous message passing instead of sharing the state between
        threads like Java does, so that the runtime itself handles concurrency;
  \item \textit{Functional}: Elixir is a pure functional language so each
        message is passed by copy, thus makes the testing easier and the
        semantics of local and remote failures the same;
  \item \textit{Actor model}: OTP fits perfectly in the actor model, since the
        state and the behaviour of an actor are neatly separated: the behaviour
        of an actor is able to change its state only after having finished to
        process a message.
        In other actor model implementations this is not always true: for
        example, in Scala an actor \textit{may suffer of concurrent
        modifications of its state};
  \item \textit{Lightweight processes}: every thread in the JVM is mapped
        one-to-one to an OS level thread, in BEAM the mapping is n-to-m using
        lightweight threads (not OS threads) and supporting true multi-threading
        through SMP where each scheduler handle a running queue of Elixir
        processes/actors. The Akka framework implements the actor model in
        Java, but since the model is not natively supported by Java the
        implementation has some important drawbacks compared to the BEAM one.
        As Akka actors run on OS threads, an actor that randomly blocks will
        block the entire thread. This can not happen in BEAM, where a blocked
        actor does not block the scheduler which immediately starts running the
        next process in its queue;
  \item \textit{Isolation}: the JVM uses a shared heap among all existing
        threads. In BEAM each Elixir process/actor has its own heap;
  \item \textit{Heap Management}: BEAM process isolation makes garbage
        collection easier and more efficient than in Java. The garbage
        collection process does not affect other running processes because
        there isn't any global pause in the application like in the JVM;
    \item \textit{Hot code swapping}: BEAM supports hot code swapping, allowing
        to update the code whithout stopping the process which is running it
        and defining procedures to employ when swapping code at runtime;
    \item \textit{Supervision trees}: OTP provides primitives to structure a
        fault-tolerant application in a hierarchical tree of processes where
        the parent can handle the faulty behaviour of the children with
        different strategies;
    \item \textit{Location transparency}: the implementation needed to invoke a
        remote procedure/send a message to a remote process is actually the same
        of using a local procedure/send a message to a local process.
\end{itemize}

\paragraph{Elixir-specific features}
\begin{itemize}
  \item \textit{Productivity}: the Mix tool enables both good short-term
        productivity as well as long-term productivity and maintainability;
  \item \textit{Extensibility and Compatibility}: the most part of the Elixir
        language is written in Elixir itself. This is because the smallest
        programming units (e.g., \texttt{if} statements) are written using
        macros which are unfolded at compile time, thus guaranteeing no loss of
        performance and smooth compatibility w.r.t. Erlang , as well as a way
        to easily define DSLs;
  \item \textit{GenStage}: from Elixir 1.4, Flow and the GenStage behavior
        (a behavior the equivalent of an abstract class in OOP) has been
        implemented.
        The GenStage behavior is built on top of GenServer (Generic
        Server, the basic OTP actor) and provides handy producer-consumer
        features.
\end{itemize}
