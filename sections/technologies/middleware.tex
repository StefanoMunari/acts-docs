\subsection{Middleware}
We have chosen Elixir (an Erlang\footnote{Erlang is a general-purpose, 
concurrent, garbage-collected programming language} dialect) to implement 
our middleware. Elixir comes with a series of features like meta programming, 
mix tool and the ExUnit test framework which will help us to make 
the project maintainable.
However, the main reason to choose Elixir/Erlang is BEAM.
BEAM is the Erlang runtime which provides an enviroment to build portable, 
distributed, fault-tolerant and soft real-time applications.
Since Erlang does not have a formal specification for its runtime we will
 base our motivations on a comparison between BEAM and the JVM 
(Hotspot implementation)\footnote{T. Pool \textit{Comparison of Erlang Runtime 
System and Java Virtual Machine}, University of Tartu, Estonia, May 2015}.

\paragraph{BEAM}
\begin{itemize}
    \item \textit{Lightweight processes}: 
    \item \textit{Isolation}:
    \item \textit{Heap Management}:
    \item \textit{Concurrency model}:
    \item \textit{Hot code swapping}:
    \item \textit{Supervision trees}: 
\end{itemize}
