\subsection{Message Broker}
We employ RabbitMQ as message broker
for the communication generated by the middleware and directed either to
the middleware of another node or to the application server.

We chose RabbitMQ over other options for several reasons:
\begin{itemize}
  \item \textit{Important pedigree}: RabbitMQ is known to be among the
        most widely deployed open source message brokers. Indeed, even if it
        is written in Erlang (an uncommon technology in
        the whole software industry), RabbitMQ is widespread as it provides
        many tools for the most popular programming languages;
  \item \textit{Easy and light}: RabbitMQ is both easy to deploy and light,
    causing very little overhead;
  \item \textit{Configurable}: RabbitMQ supports several messaging protocols.
        Nevertheless, we will use the (common) AMQP 0-9-1 protocol
        \cite{vinoski2006advanced};
  \item \textit{Simple to integrate}: as said before, RabbitMQ is written in
        Erlang and therefore it is straightforward to integrate it in an Elixir
        system.
        Other than this, there has been recent work in the Elixir community to
        provide wrappers for RabbitMQ, with Wabbit among these. Wabbit
        is a RabbitMQ adapter which is built on top of GenStage, allowing the
        programmer to just think of how to process an incoming stream of
        messages.
\end{itemize}
