\subsection{Application Server}\label{sec:sol-des-as}

The PEAP stack structures the Phoenix application as an Umbrella application
composed of two services: \texttt{Domain} and \texttt{Interface}.
\texttt{Domain} is intended to collect and represent information about the
application, and also to provide a means to access a database.
\texttt{Interface} is accountable for managing all the networking I/O
interactions of the whole Phoenix application, so it will receive data from the
backend and serve it to the end users.

\subsubsection{Application Server -- Domain}
%TODO: Update if something changes

The \texttt{Domain} application keeps track of per-district data.
Clearly, it would be too expensive to store fine-grained information, so we
decided to store basic information in each tracker, for instance the number of
streets (\textit{static} information) and the number of travelers
(\textit{dynamic} information).

\subsubsection{Application Server -- Interface}
In the \texttt{Interface} application we leverage the Phoenix pub/sub
mechanism in order to deliver live information to consumers.
Basically, we make end users register their sockets to one or more topics by
means of a Phoenix Channel and then we deliver to them information which
matches those topics.

\texttt{Interface} also exposes some REST APIs through which it is possible to
fetch data about cities, districts, and pages.
