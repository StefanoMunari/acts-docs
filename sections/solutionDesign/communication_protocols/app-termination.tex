\subsubsubsection{Termination}
When describing the shutdown of the whole system, we assumed that the
application terminates gracefully.
In this section we show the algorithm that is used to achieve this goal.

As we can see from figure \ref{fig:app-proc-tree}, the termination follows the
opposite order as the bootstrap.

\begin{enumerate}
  \item The \textit{Master} task $M$ stops active entities, e.g. pedestrians;
  \item $M$ saves into a file the state of the active entities once it has
    stopped all of them;
  \item $M$ saves into a file the state of reactive entities. It is important to
    notice that is now safe to save their internal state, since no active
    entities can modify it anymore;
  \item $M$ sends a \texttt{app\_shut} message to the middleware;
  \item $M$ terminates itself and the entire application therefore stops.
\end{enumerate}

The middleware layer can request the application to stop via the
\texttt{app.shutdown} call.

The last operation the \textit{Master} task does is to send a message to inform
the middleware layer of the successful termination.
Similarly to the bootstrap phase, the middleware expects to receive this
message within a certain amount of time. If this timeout expires, the
middleware layer calls again the \texttt{app.shutdown} procedure exposed by the
application.
