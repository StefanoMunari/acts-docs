\subsubsection{Artificial Intelligence (AI)}

To solve the path
problem outlined in Section \ref{sec:pa-distribution}
we have implemented a path-finder for the agents of our system.
Our AI is written in C++ but we have written the Ada bindings for it.
Hence, we are able to statically link the AI to our project.

The path-finder is executed by the ACTS scheduler. In particular, a
worker runs the \verb|Agent.Act| procedure which can contain invocations
to the AI.

% The AI needs to know on which graph/infrastructure the agent
% is traveling on. Indeed, different travelers moves on different
% infrastructures. For example, a car needs to travel only on roads which are
% mostly
% separated from other city infrastructures. Thus, the AI maintains three
% search graphs, one for each infrastructure (roads, bicycle paths, sidewalks).
% In case of intersection or crossing stretches, the ids (of the stretches)
% are replicated over different graphs.


\subsubsubsection{Environment}

The first step when planning an AI consists in analyzing the task environment
to clearly understand the problem:

\begin{itemize}

\item{deterministic} --
the next state is completely defined by the previous state plus the move
performed by the agent;

\item{discrete} --
there are many states of the system, but their number is finite.
The agent acts in a discrete space and the information which he manipulates
are also discrete (e.g., costs of moves);

\item{sequential} --
the choice of the next move depends on the previous one. The agent creates
a totally ordered sequence of moves;

\item{dynamic} --
the environment change around the agent while it is computing its moves;

\item{multi-agent} --
there are several agents who act concurrently in the same virtual
environment;

\item{partially observable} --
since the system is distributed, the agent does not have access to the complete
information about the whole environment. At any time, an agent is
traveling on a
specific district which has only updated information on its state and does not
know the costs of the other districts. Also, an agent eventually needs to
compute a path which comprises several districts.
\end{itemize}

\subsubsubsection{Search algorithms}

Due to time constraints related to the path computation, we preferred to
choose a not-real-time solution. However, we recognize that a real-time solution
helped by a traffic forecast model would be more precise and intriguing.
Our solution uses classical path-finding algorithms. These algorithms
belong to uninformed and informed search classes.

Our AI offers three path finding algorithms:
\begin{itemize}
  \item Greedy Search;
  \item Uniform Cost Search;
  \item A* Search.
\end{itemize}

The first one is incomplete and not optimal because of the local search which
looks only one step ahead (visiting
only the neighbors). The last two algorithm are both complete and optimal
under certain assumptions. They have been defined
to guarantee our agents to always find the best possible path with
the minimum number of visited nodes in the search graph.

\subsubsubsection{Properties}

To guarantee completeness we need to assume that each graph is a strongly
connected component. Basically, it means that there are no isolated
components. Thus, the agent knows there will always be at least a path between
the source and the destination of its plan. % OK its, usiamo sempre neutrum per agent

Furthermore, our AI considers non-grid based maps: which means some common
heuristics are not applicable a priori (e.g. Manhattan distance, euclidean
distance). Also, since grid-based maps are a subset of all possible maps, we
can say that our AI is independent on the map topology.

To solve the path finding problem we reason about the specific domain and its
abstractions in terms of infrastructure building blocks and travelers. We
found an admissible and consistent heuristic for A* and, more in general, other
informed algorithms. To concretely guarantee the assumptions hold for all the
graph configurations, we apply the Tarjan algorithm to each of them.

Tarjan automatically validates our graphs at build time by ensuring that there
is only one strong connected component for each graph. Otherwise, it reports
an error and the isolated component.
Finally, the AI is agent-independent, which means we can use it to calculate
the path of different kinds of agents: bicycles which move on bicycle lanes,
pedestrians who move on sidewalks and motor vehicles which move on roads.
Thus, each infrastructure is represented by a different infrastructure graph.
Overall, the AI manipulates three infrastructure graphs plus N costs graphs
which represent the traffic costs of a specific district.

\subsubsubsection{Heuristic}

We have approximated the dynamic component of the environment by considering
the state of the system during the simulation as a sequence of snapshots.
Thus, each snapshot can give to the path-finder the information it needs
about the city traffic.
This information is detected by a daemon process which belongs to the AI.
It uploads new generated graphs of the dynamic costs (the costs of the traffic)
in the AI graph registry. The daemon action is triggered each time a new
snapshot is generated.

The traffic costs are computed as the number of agents which are on a specific
stretch when the snapshot is performed. Also, the intersections have an additional
cost given by the traffic lights which can slow down the travel of the agent.

Unfortunately, the system
snapshots were not provided as planned at the beginning of the project, so
the AI computes paths always on the same infrastructure graph without
considering the traffic costs. However, its effectiveness (with traffic costs)
has been proved by running it on a city written in NodeJS for another
proof of concept simulator.
