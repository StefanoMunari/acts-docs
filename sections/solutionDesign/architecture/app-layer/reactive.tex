\subsubsubsection{Reactive entities}
This is one of the largest packages of our system.
It contains:
\begin{itemize}
	\item all the reactive entities (streets, lanes, stretches, etc.);
	\item the builders and factories of the reactive entities;
	\item the district class, which acts as a Facade for
the whole application (except for scheduling);
	\item the entity registries and directories (subdivided by entity types);
	\item some scaffolding packages like Utils.
\end{itemize}

The important interfaces of the package are \texttt{Infrastructure} and
\texttt{Treadable}. The former represents an infrastructural object while the
latter represents a piece of infrastructure on which an agent can travel
(i.e. stretches and intersections).
In our system we consider the streets and lanes just as containers
implemented using the \textit{Composite} pattern:
\begin{itemize}
	\item a \texttt{Street} is a composition of \texttt{Way}s;
	\item a \texttt{Way} is a composition of \texttt{Lane}s;
	\item a  \texttt{Lane} is a composition of \texttt{Stretch}es.
\end{itemize}

Thus, we consider the stretch as the basic unit
of the street where an agent can travel.
Handling the concurrency at this level potentially enables
an high level of  parallelism in our system.

The other basic infrastructural entity on which an agent can travel is
\texttt{Intersection}. Each \texttt{Intersection} reserves one of its
entries for a traveler if it is a vehicle, otherwise it is just bypassed by
pedestrians and bicycles.

A \texttt{Treadable} object do not make worker threads block in its
critical regions.
Any time a traveler moves on them, a \texttt{Treadable} object attempt
to make it enter. If this
operation is successful, the traveler advances in the next piece of
infrastructure
and leaves the former one. Otherwise, the traveler is rescheduled
for a time span ahead
to retry the operation. In the latter case, the traveler does not leave
the reactive entity where he is currently in.

In this package we also implemented the decorators for stretches and lanes: they
override basic operations by composition and enrich some of them with custom
behavior. For example:

\begin{itemize}
  \item before entering a new lane it may be applied a given speed limit;
  \item before treading a stretch, further checks may be required if it is a
    part of a pedestrian/bicycle crossing.
 \end{itemize}
