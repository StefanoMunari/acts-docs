\subsubsubsection{Boot}\label{sec:mw-boot-descr}

The Boot application is responsible
of gracefully starting the distributed
system.
It does so by dividing the start process in two steps, namely the
\textbf{marker diffusion} and its \textbf{(own) boot}. The procedure is
the same shown in Figure \ref{fig:sys-bootstrap-protocol}: the first step
is described by reading the figure from left to right
while the second one can be seen reading the same figure
in the opposite direction.
\\

There are two possible logical markers which have different semantics:
\begin{itemize}
	\item \textit{boot marker} : represents a boot request. The node which send
	a boot marker has previously received a boot marker or
	is the master node of the system which is starting the boot process.
	\item \textit{end marker} : represents a reply to a boot marker. A node
	can reply with an end marker only if all the boot markers of the node have
	been satisfied with an end marker plus the node itself has started its
	application layer successfully. Basically an end marker reply
	means that the (sender) node
	has complete the boot of its subtree of nodes.
\end{itemize}

When receiving a marker, the Boot application forwards the marker to all
of its adjacent middleware nodes which have not yet sent a boot request to it.

If the node receive further boot markers after the first one, then it simply
ignores them because it is already proceeding with the boot.

After having received all the end marker replies,
the Boot application will start the application layer
by sending a boot request to it.
When the application layer replies with an ack, the node has complete its boot
and it replies with an end marker to all the middleware nodes which have
previously sent a boot marker.
