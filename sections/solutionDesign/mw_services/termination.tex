\subsubsection{Termination service}

The Termination service is responsible of shutting down the system gracefully.
It does so in a way very similar to what we have seen in Section
\ref{sec:mw-boot-descr} for the Boot service, with a few differences in the
first of the two phases.

Initially, this service is in a \texttt{running} state, which means the
application layer is normally running.

If the middleware receives a stop marker, it will send an asynchronous request
to the application to stop and it will forward the stop marker to all of its
neighbors (any of the subsequent incoming request markers will receive an
immediate reply).

As soon as the application will inform the middleware layer that the former is
stopped, the latter will enter a \texttt{stopped} state. Then, when the
Termination service receives all the replies for the stop marker from its
neighbors, it will send a reply back for the stop request marker and enter a
\texttt{terminating} state.

During this second phase, the middleware will wait for a termination marker.
When it arrives, it will forward the termination request marker to all of its
neighbors and reply immediately to any other termination request marker which
arrives.

When all of the neighbors of a node have sent back a reply to the termination
marker, the node will ask the application layer to terminate and will sent a
reply for the termination request back, ending the termination procedure and
entering a \texttt{terminated} state.
