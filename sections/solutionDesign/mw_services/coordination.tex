\subsubsection{Coordination service}

This component is responsible of the coordinating middleware nodes by
providing mechanisms with which the system is able to operate cohesively.

We show in figure \ref{fig:mw-coordination} the architecture of this service
and then we will show in detail each module that composes this component.

%\begin{figure}[H]
%  \centering
%  \includegraphics[width=\columnwidth]{images/solution/mw/coordination.eps}
%  \caption{Middleware's Coordination service}
%  \label{fig:mw-coordination}
%\end{figure}
% TODO: Add diagram

% TODO: All class diagrams have to be added
\subsubsubsection{coordination.Coordination}
% TODO: Class diagram
\FloatBarrier
\begin{itemize}
  \item \textbf{Description} \\
    This module is the Fa\c cade of the Coordination service. It is responsible
    to boot neatly and supervise all processes in Coordination.
  \item \textbf{Attributes}
  \item \textbf{Operations}
  \begin{itemize}
    \item \texttt{+ start()} \\
    Starts the Coordination service.
    \item \texttt{+ handleMessage(message: String)} \\
    % TODO: check this out: Message will really be a String?
    Receives a message and delegates the proper handling of it to an internal
    module.
  \end{itemize}
\end{itemize}

\subsubsubsection{coordination.coordinationInfo}
% TODO: Class diagram
\FloatBarrier
\begin{itemize}
  \item \textbf{Description} \\
    Process that holds the current state of the coordination.
  \item \textbf{Attributes}
    \begin{itemize}
      \item \texttt{- state: Map<Atom,Any>} \\
    Information about coordination.
    \end{itemize}
  \item \textbf{Operations}
  \begin{itemize}
    \item \texttt{+ startLink()} \\
    Starts the process and initialize \texttt{state} as an empty map.
    \item \texttt{+ getInfo(key: Atom)} \\
    Returns the information contained in the state at the entry \texttt{key}.
    \item \texttt{+ setInfo(key: Atom, value: Any)} \\
    Modifies the state held by the process by setting the entry \texttt{key}
    to the value \texttt{value}.
  \end{itemize}
\end{itemize}

\subsubsubsection{coordination.election.Election}
% TODO: Class diagram
% TODO: Explode class in interface and implementation
\FloatBarrier
\begin{itemize}
  \item \textbf{Description} \\
    Process that is responsible to manage the election of a system coordinator.
  \item \textbf{Attributes}
    % TODO
  \item \textbf{Operations}
    % TODO
\end{itemize}

\subsubsubsection{coordination.time.HeartbeatMaker}
% TODO: Class diagram
\FloatBarrier
\begin{itemize}
  \item \textbf{Description} \\
    Process that has to produce an heartbeat (i.e., a signal that represents
    the logical clock ticking in the system) for all the nodes in the
    system.
  \item \textbf{Attributes}
  \item \textbf{Operations}
  \begin{itemize}
    \item \texttt{+ start()} \\
    Starts HeartbeatMaker process.
    \item \texttt{+ enable()} \\
    Makes HeartbeatMaker begin producing heartbeats.
    \item \texttt{+ disable()} \\
    Makes HeartbeatMaker stop producing heartbeats.
  \end{itemize}
\end{itemize}

\subsubsubsection{coordination.time.HeartbeatListener}
% TODO: Class diagram
\FloatBarrier
\begin{itemize}
  \item \textbf{Description} \\
    Process that listens to heartbeats. It is responsible to delivering them
    to the application layer.
  \item \textbf{Attributes}
  \item \textbf{Operations}
  \begin{itemize}
    \item \texttt{+ start()} \\
    Starts HeartbeatListener process.
    \item \texttt{+ beat()} \\
    Notifies the application layer of a new heartbeat.
  \end{itemize}
\end{itemize}
