\subsubsubsection{Reactive entities}
This is one of the largest packages of our system: it contains all the types of
reactive entities (streets, lanes, stretches, etc.) as well as their builders
and factories, the district class (which acts more or less as a controller for
the system), the entity registries and directories (subdivided by entity types),
and finally some scaffolding like Utils packages.

Here the most important interfaces are \texttt{Infrastructure} and
\texttt{Treadable}, which respectively represents an infrastructural object and
a piece of infrastructure which is treadable, that is stretches and
intersections.
In fact, even if streets and lanes would actually be treadable (in the ``real''
world), in our system we will think of them just as containers which use the
\textit{Composite} pattern: a \texttt{Street} is a composition of \texttt{Way}s,
which in turn are composed of \texttt{Lane}s, which finally are composed of
\texttt{Stretch}es.
For the sake of ease, we will think of stretches as the basic treadable unit
for streets, handling the concurrency at this tiny architectural level (thus
enabling a high level of potential parallelism in our system).

The other treadable entity is \texttt{Intersection}: they reserve one of its
entries for a traveller if it is a vehicle, otherwise it is just bypassed by
pedestrians and bicycles.

Treadable objects do not make worker threads block in critical regions: each
time a traveller moves on them, treadables attempt to make it enter. If this
operation is successful, travellers advance in the next piece of infrastructure
and leave the former one, otherwise they are rescheduled for a little time ahead
to retry the operation (without leaving the reactive entity they are currently
in).

In this package we can find also the decorators for stretches and lanes: they
override basic operations by composition and enrich some of them with custom
behaviour: for example:
\begin{itemize}
  \item before entering a new lane it may be applied a given speed limit;
  \item before treading a stretch, further checks may be required if it is a
    part of a pedestrian/bicycle crossing.
 \end{itemize}
