\subsubsection{Solution to Concurrency Problems}

Our solution addresses the problems we pointed out in section 
\ref{sec:pa-concurrency}.
Therefore here we are providing answers to the questions 
we made ourselves at the beginning of the development.

\paragraph{How can we recognize that a situation is suitable to present
concurrency?}
There are several active entities in our application: 
pedestrians, cars, bikes and buses. 
They act because of their own initiative, 
so they are likely to be considered as active tasks.

Moreover, their actions cause side-effects to reactive entities like roads.
Consequently, our application does not only need to be parallel 
but also \textit{concurrent}.

\paragraph{How concurrent accesses will be managed? Will it make sense to
discern side-effect accesses (procedures) from read-only accesses
(\textit{functions})?} 
Concurrent accesses will be managed by means of \textit{Protected Objects}, 
i.e. resources that can be accessed in concurrent read
\textbf{or} mutual exclusive write mode.

It indeed makes sense to have the mentioned separation. 
It may be useful to modify roads' status when vehicles travel through them. 
Furthermore it might be useful to get information from them 
when someone asks for it (without causing side-effects).

\paragraph{Which entities have to be active?} 
In light of active entities undertake spontaneous actions, in our context they are:
\begin{itemize}
  \item Moving entities, which move around the city according to their own will;
  \item Semaphores, which autonomously changes its color by the time elapses.
\end{itemize}

\paragraph{Which entities have to be reactive?} 
Roads (with all their parts) and crossroads should be reactive entities:
\begin{itemize}
  \item A road is used to travel towards a destination 
  and it takes actions only when it is trodden. 
  However, the state of a road is composed of the states 
  of the reactive sub-entities it comprises, i.e. stretches and houses;
  \item A Crossroads coordinates the traffic at road intersections 
  exclusively whenever a moving entity requires to cross it.
\end{itemize}

\paragraph{Which entities have to be passive?} 
Passive entities can not alter its internal state. 
Therefore, road signs fit perfectly in this definition, since
they represent some immutable information that is read by road users.

\paragraph{Should we use any formal technique to proof the correctness 
of the system properties relating to concurrence?} 
We will strive to motivate the correctness of our application 
but we deliberately do not choose any specific technique.

\paragraph{Do we have the possibility to incur a non-valid state? 
If so, how do we handle the problem?} 
The application will be designed to avoid non-valid states. 
However, if the system recognizes it is in a non-valid state, 
it recoveries the execution from the last valid snapshot.

\paragraph{How can we avoid possible starvation scenarios?} 
The application will be designed to overcome starvation scenarios.

\subsubsection{Solution to Domain Problems}

In section \ref{sec:pa-domain-problems} some problems 
inherently related to the application domain are explained, 
and our solution design is thought to solve them.

\paragraph{Pedestrian Deadlock}
The system will be able to detect such a scenario
and it will solve the problem by moving on
all involved pedestrians together.

\paragraph{Rear-end Collisions} 
If the next road stretch of a moving entity route is already occupied,
the entity will wait for the stretch to become free before moving on.

\paragraph{Stretch Parallelism} 
One road stretch (e.g. a sidewalk stetch) may contain more 
than one entity of the same type (e.g. pedestrians) at a time.
Hence, it is necessary to introduce a counter which represents
the number of entities currently treading the stretch.
It is indeed useful for denying the possibility to enter a stretch if it is ``full''.

\paragraph{Yield rules} 
The logic to manage yield rules will be encapsulated in crossroads, 
which has to ensure the respect of the street code.
