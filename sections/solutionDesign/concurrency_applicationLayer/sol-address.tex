%%%%%%%%%%%%%%%%%%%%%%%%%%%%%%%%
%% CONCURRENCY
%%%%%%%%%%%%%%%%%%%%%%%%%%%%%%%%

\subsubsection{Concurrency Problems}
\label{sec:sol-conc-problems}

Our solution addresses the problems we pointed out in Section
\ref{sec:pa-concurrency}.
Therefore here we are providing details on how we tackled the questions we
made ourselves at the beginning of the development.

\paragraph{How can we recognize that a situation is suitable to present
concurrency?}
There are several active entities in our application: pedestrians, cars, bikes
and buses. They act because of their own initiative, so they are likely to be
considered as active tasks.

Moreover, their actions cause side-effects to reactive entities like roads.
Consequently our application needs to embody concurrency, i.e., potential
parallelism.

\paragraph{How concurrent accesses will be managed? Will it make sense to
discern side-effect accesses (procedures) from read-only accesses
(\textit{functions})?}
Concurrent accesses will be managed by means of \textit{Protected Objects},
i.e., resources that can be accessed in concurrent read \textbf{or} mutual
exclusive write mode.

Indeed, it is reasonable to have the mentioned separation.
It may be useful to modify roads' status when vehicles travel over them.
On the other hand, it might be useful to get information from them when
someone asks for it without causing side-effects, and this should be
doable in
parallel.

\paragraph{Should we use any formal technique to proof the correctness
of the system properties relating to concurrence?}
We will strive to motivate the correctness of our application but we
deliberately do not choose any specific technique.
We did some concurrent tests, that is scheduling tests. In these tests we prove
that executors can start, distribute work across several worker threads and
stop gracefully.

\paragraph{Do we have the possibility to incur a non-valid state?
If so, how do we handle the problem?}
The application will be designed to avoid non-valid states.
However, if the system recognizes it is in a non-valid state, it resumes the
execution from the last valid snapshot.
% resumes: see https://www.cs.york.ac.uk/rts/books/RTSbookThirdEdition/chap6.pdf

\paragraph{How can we avoid possible starvation scenarios?}
The application will be designed to overcome starvation or deadlock scenarios.

In fact, active entities do not hold locks on protected objects and do not
block on Remote Procedure Calls (or \textit{RPC}s): while the first condition
is quite easy to achieve (the Ada runtime helps), the latter
has been not straightforward to implement.

%%%%%%%%%%%%%%%%%%%%%%%%%%%%%%%%
%% APPLICATION
%%%%%%%%%%%%%%%%%%%%%%%%%%%%%%%%

\subsubsection{Application Logic Problems}

\paragraph{Pedestrian Deadlock}
Actually, our system does not have this problem. We organize sidewalks in
\textit{lanes}. Though sidewalks in real world usually do not have lanes (apart
from seldom exceptions), we employ the concept of a \textit{logic} lane, i.e.
an abstraction that allows us to easily
coordinate opposite flows of pedestrians.

Since pedestrians proceeding in opposite directions will not compete to enter
the same sidewalk stretch, there can not be any pedestrian deadlock.

We re-applied the same mechanism to crossings
(both for pedestrians and bikes): we
assign an arbitrary direction at initialization time, but we do it
\textit{only for the topology file we feed the AI with}: the backend is
completely unaware of the fact a crossing may have a direction.

\paragraph{Rear-end Collisions}
If the next road stretch of a moving entity route is already taken, the entity
will wait for the stretch to become free before moving on.
This is done by regulating concurrent accesses to a stretch internal state via
a protected object; as we said before, we don't want our worker threads to
block on entries, therefore if the stretch is taken then a worker thread will
be rescheduled for a later moment in time to check again.

\paragraph{Stretch Parallelism}
One road stretch (e.g., a sidewalk stetch) may contain more than one entity of
the same type (e.g., pedestrians) at a time.
Parallelism is achieved by using a system of worker threads thanks to which
more than one active entity can potentially act in parallel in a district.
Other than this, there is potential parallelism since locks are not held on the
whole street (/way/lane), but only on the stretch a traveler is willing to
tread.

\paragraph{Yield rules}
The logic to manage yield rules will be encapsulated in crossroads, which has
to ensure the respect of the street code.

\paragraph{Booking a park spot}
A traveler which is going out of a building with a vehicle will try to book
beforehand a spot in the destination building garage (which has limited
capacity), so that when she'll arrives at destination she will find room to
park her vehicle.

Even if this may not seem a concurrency problem at a first glance, it is
indeed: two travelers may be leaving from each other's destination, and try
to book a spot in the garages they are leaving.
Therefore, if they both retain a lock on the garage to hold the vehicle they
are coming out with, they are exposed to a deadlock hazard.

We address this issue by making the garage implementation thread-safe, namely
having it be composed of three protected objects:

\begin{itemize}
  \item \textbf{Parked vehicles:} vehicles which are parked in the garage
    without anyone onboard
  \item \textbf{Leaving vehicles:} vehicles which are currently exiting the
    garage
  \item \textbf{Pending vehicles:} vehicles which are not yet arrived in the
    garage or that are currently booking a park spot
\end{itemize}

Thanks to this division, we are able to avoid holding a lock on the garage
object and access its state concurrently without blocking inside the critical
regions.

\paragraph{Zebra Crossings} \mbox{} \\
Zebra crossings grants the right of being crossed with a higher priority by
\textit{privileged} entities: in the case of pedestrian crossings, these
entities are (obviously) pedestrians, meanwhile bikes are the privileged
entities in the case of bike crossings.

Whenever a privileged entity attempts to tread a crossing stretch, it is
inserted in a list. If an unprivileged entity tries to enter the same stretch
after a privileged entity was inserted in the queue, it will not be allowed to
tread the stretch, meanwhile another privileged entity might enter the stretch
under the same conditions (and if there is enough room).
Hence, unprivileged travelers will be able to tread zebra crossing stretches
only if they enter them \textit{before} a privileged entity performs the same
request.


%%%%%%%%%%%%%%%%%%%%%%%%%%%%%%%%
%% TIME
%%%%%%%%%%%%%%%%%%%%%%%%%%%%%%%%

\subsubsection{Time Problems}

In Section \ref{sec:pa-time-problems} we point out some questions
about time management in our simulations; hereafter we provide
details on how we implemented the time management in our system.

\paragraph{How can we simulate road users proceeding (considering the
travelling speeds)?}
Our system will be comprehensive of a scheduler, which executes actions:
active entities will implement a common interface (\texttt{Active.Agent})
through which the scheduler is able to run them.

Actions are enqueued by using the package \texttt{Ada.Real\_Time} and one-shot
timers: when each timer expires, a callback will be responsible of pushing the
action on a queue consumed by a set of worker threads.

Active entities will therefore be rescheduled with a delay which is inversely
proportional to their speed. We will use a minimum time period for the maximum
speed achievable and calculate the deferral time for the next action by
dividing the speed of a traveler by this value.

\paragraph{How can we simulate crossroads (considering road users' arrivals
  time)?}
Crossroads logic will be managed by means of an Ada Protected Object. By
carefully handling concurrent accesses to these objects, we aim to provide the
exact semantic for intersections between roads.

Other than this, each crossroad is regulated by a set of traffic lights. They
change color as the time flows, and this causes the intersection to activate
only some of the exits from time to time.


\paragraph{How are semaphores going to be able to synchronize? Which of them
  decide how much time do they have to wait for?}
Semaphores are active entities, and therefore their actions will be
periodically scheduled.
The period of a semaphore is a configuration parameter.

\paragraph{How long do people stay in facilities for? How much precise do these
  intervals have to be?}
Time spent in facilities is decided arbitrarily as a fixed configuration
parameter.


\paragraph{When is it suitable to use logical clocks instead of local system
  clock?}
We use the system clock of the operating system to
achieve real concurrency between nodes. Indeed, if an entity is scheduled for a
given time, it should not wait to be synchronized
in order to execute. Initially, we implemented a logical clock
based scheduler which indeed showed the described problem.
Successively, after recognizing the error, we have adopted an OS based approach
which helps also the whole system to scale better
locally.
