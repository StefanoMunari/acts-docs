\subsubsection{Solution to Concurrency Problems}

Our solution addresses the problems we pointed out in section
\ref{sec:pa-concurrency}: here we are providing answers to the questions we
made ourselves at the beginning of the development.

\paragraph{How can we recognize that a situation is suitable to present
concurrency?} There are several active entities in our application: pedestrians,
cars, bikes and buses. They act because of their own initiative, so they are
likely to be considered as active tasks.

Moreover, their actions cause side-effects to reactive entities like roads.
Hence, from this situation we understood that our application does not only
need to be parallel but also \textit{concurrent}.

\paragraph{How concurrent accesses will be managed? Will it make sense to
discern side-effect accesses (procedures) from read-only accesses
(\textit{functions})?} Concurrent accesses will be managed by means of
Protected Objects, i.e. resources that can be accessed in concurrent read
\textbf{or} mutual exclusive write mode.

It indeed makes sense to have this separation of concerns: it may be useful to
modify roads' status when vehicles travel through them, but it also might be
useful to get information from them when someone asks for it (without causing
side-effects).

\paragraph{Which entities have to be active?} As stated before, moving entities
and semaphores should be active since they act because of their own initiative:
\begin{itemize}
  \item Moving entities travels through the city and they do it because of
    their will;
  \item A semaphore changes its color by the time elapses.
\end{itemize}

\paragraph{Which entities have to be reactive?} We think that roads
(their parts are reactive too) and crossroads should be reactive entities:
\begin{itemize}
  \item Roads are used to travel to the destination, therefore they do not take
    action if not trodden. However, the state of a road is composed of the
    internal states of the reactive sub-entities it comprises, i.e. stretches
    and houses;
  \item A crossroads coordinates the traffic at intersections only when there
    are moving entities that require to cross it.
\end{itemize}

\paragraph{Which entities have to be passive?} Passive resources do not have
internal state. Therefore, road signals fit perfectly in this definition, since
they represent some immutable information that is read by road users.

\paragraph{Should we use a formal technique to give proof of deadlock/livelock
avoidance?} We will explain at the best of our possibility why the application
will avoid deadlocks or livelocks.

\paragraph{Should we use a formal technique to give proof of some properties of
the system that depend on concurrency?} Also for this point, we will strive to
motivate the correctness of our application but we deliberately do not choose
any specific technique.

\paragraph{Do we have the possibility to incur a non-valid state? If so, how do
we handle the problem?} The application will be designed to avoid non-valid
states. However, if the system recognizes it is in a non-valid state, it resume
the execution from the last valid snapshot.

\paragraph{How can we avoid possible starvation scenarios?} The application
will be designed to overcome starvation scenarios.

\subsubsection{Solution to Domain Problems}

In section \ref{sec:pa-domain-problems} we showed that the applicative domain
presents inherently some problems, but our solution design is thought to solve
them.

\paragraph{Pedestrian Deadlock} After detecting this scenario with a bit of
application logic, the system will solve the situation by moving at the same
time pedestrians to the start of the next road stretch.

\paragraph{Rear-end Collisions} If the next road stretch is already occupied,
then the moving entity will wait for the stretch to become free while keeping
its own stretch occupied.

\paragraph{Stretch Parallelism} Sidewalks may contain more than one pedestrian.
As opposed to roadway lanes, it is sufficient to add a counter which represents
the number of entities currently treading the stretch (and denying the
possibility to enter the stretch if it is ``full''.

\paragraph{Yield rules} The logic to manage yield rules will be encapsulated in
crossroads, which will make vehicles respect the rules without letting anyone of
them violating the street code.
