\subsubsection{Semaphore}
A semaphore is placed at the end of a road stretch.
A road stretch with a semaphore exposes the following functional channels:
\begin{itemize}
	\item \texttt{stop(way)}
	\begin{itemize}
		\item \textit{description}: it blocks outgoing flow of vehicles going towards a given way
	\end{itemize}
	\item \texttt{yield(way)}
	\begin{itemize}
		\item \textit{description}: it slows outgoing flow of vehicles going towards a given way
		\item \textit{details}: Non esiste a livello applicazione: è un rosso ma a livello grafico si può rappresentare con giallo
	\end{itemize}
	\item \texttt{go(way)}
	\begin{itemize}
		\item \textit{description}: it allows outgoing flow of vehicles going towards a given way
	\end{itemize}
\end{itemize}
Notes:
\texttt{stop}, \texttt{yield} and \texttt{go} are likely to be called by a set of semaphores.
A set of semaphores communicate internally to advance through their RGY steps.
way parameter might be N/S (“vertical” street) or W/E (“horizontal” street).