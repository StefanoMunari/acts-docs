\subsection{Broker}\label{sec:impl-broker}

\paragraph{Achieving transparency}
% In order to have a glimpse of the reason why we say the application is
% transparent to the middleware and viceversa, here we will make an example which
% involves the broker subsystem.

When travelers manage to tread a new piece of infrastructure, they publish an
event by issuing an asynchronous request to the middleware. However, they do it
without knowing (or specifying) if the tread operation was local or remote,
because (despite the qualified name of the Callbacks package) travelers just
state a Tread operation was completed.
For what concers the middleware, it sees an asynchronous request named
``TREAD''. Since the middleware does not know anything about the application
logic domain, it just recognizes an asynchronous event was issued by the
application layer and forwards the message to our broker.
The broker subsystem is instead able to distinguish if a movement was local or
remote by means of a little hack:

\begin{itemize}
  \item the application layer states the operation involves an entity which can
  drive or walk, and it refers to it by its logical identifier;
  \item upon the message reception, the middleware looks up for the recipient
    node by performing name resolution on the identifier specified in the
    message headers;
  \item the middleware then appends the recipient and sender fields to the
    (middleware-layer) message, which gets forwarded to the broker. We do not
    lose transparency, since the recipient and sender fields carry information
    about the backend nodes involved in the operation;
  \item if the broker reads two different values for the sender and the
    recipient fields, then it knows this is a remote operation.
    Otherwise, in the
    case the sender and the recipient fields are equal, it knows a traveler did
    not move to another district.
\end{itemize}
