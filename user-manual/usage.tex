\section{Usage}

\subsection{Simulation Streaming}\label{sec:usage-stream}

Once connected to \texttt{0.0.0.0:4005} (the address provided in Section
\ref{sec:sys-run}), you will access the ACTS dashboard. In this window, you
will find different boxes, each one reporting a name of a city.
After clicking on one of those, it will expand in a panel which is
comprehensive of the districts a city is split in.
One district has several streets, road intersections and other infrastructural
elements like buildings.

If you click on a district, you will be redirected to a Web page in which
there is an SVG canvas on which the district will be drawn.
This canvas is resizable (use the mouse scroll), draggable (drag with your
mouse to move the view) and also has buttons on the bottom right corner to
zoom in (\texttt{+}), zoom out (\texttt{-}) and reset (\texttt{RESET}) the
canvas position and zoom.

When you open a district view, this will cause the browser to require the
streaming of a simulation for the city the district belongs to: therefore, a
user will implicitly either (1) trigger the boot of a city simulation or (2)
connect to an on-going city simulation.
This is completely transparent to the user, which will just be served with a
simulation for.

\subsection{Changing District}

If you want to change the district you are viewing, you can do this by just
clicking the home icon in the top left corner and repeating the steps indicated
in Section \ref{sec:usage-stream}.
Exiting the district view will not cause the simulation end, which is instead
triggered by an internal time-out. Therefore, if you connect to the same
district twice, you will re-subscribe to the same simulation you were
previously watching.

\subsection{Interactivity}
We also have live statistics for our districts: you can find them either by
hovering above buildings and with the \texttt{Stats} button in the bottom left
corner.
