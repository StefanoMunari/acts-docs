\section{Frequently Asked Questions}

\begin{itemize}

\item How to set up a docker swarm?
\\

To set a local swam master for the simulation:

docker swarm init --advertise-addr <YOUR\_LOCAL\_IP\_ADDR>

More information can be found in the \href{https://docs.docker.com/engine/swarm/swarm-tutorial/create-swarm/}{docker swarm documentation}.

\item How much time does swarm takes to deploy the whole system?
\\

It depends on the number of cities and on the size of them\footnote{Obviously,
it depends also on the computational resources of your machine}.
The estimated time with 2 cities, each one
composed by 3 districts, is around 2-3 minutes.
A small sized system, like
the one described, is composed by almost 30 containers.
However, it depends on the number or replicas declared in the configuration files.
A replica is a duplicated container used by docker swarm to guarantee some
degree of availability of the service. This enable to alleviate the single
point of failure problem at runtime. Indeed, if a container crashes, a copy
is ready to immediately replace it while docker swarm is restarting the crashed
one.

\end{itemize}
