\section{Frequently Asked Questions}

\begin{itemize}

\item \textbf{How can I use ACTS?}

ACTS provides three basic commands:
\begin{itemize}
	\item \texttt{build}
	\item \texttt{run}
	\item \texttt{clean}
\end{itemize}
You can learn how to use them by typing the specific command in your CLI or
by typing:
\begin{lstlisting}[language=bash]
<command_name> help
\end{lstlisting}
Each one will print to stdout an explanation of its usage.

More detailed instructions are provided in Section \ref{sec:sys}.

\item \textbf{How to set up a Docker Swarm?}

To set a local swam master for the simulation:

\begin{lstlisting}[language=bash]
docker swarm init --advertise-addr <YOUR_LOCAL_IP_ADDRESS>
\end{lstlisting}

More information can be found in the \href{https://docs.docker.com/engine/swarm/swarm-tutorial/create-swarm/}{docker swarm documentation}.
Also, you can check if the Docker Swarm started by following the instructions
of Section \ref{sec:sys-swarm}.

\item \textbf{How much time does Swarm takes to deploy the whole system?}

It depends on the number of cities and on the size of them\footnote{Clearly,
it is heavily affected by the computational resources of your machine}.
The estimated time with 2 cities, each one
composed by 3 districts, is around 5 minutes.
A small sized system, like
the one described, is composed by almost 30 containers.
However, it depends on the number or replicas declared in the configuration
files.
A replica is a duplicated container used by Docker Swarm to guarantee some
degree of availability of the service. This enables us to alleviate the single
point of failure problem at runtime. Indeed, if a container crashes, another
worker is ready to immediately replace it while Docker Swarm managers restart
the failing container.

\item \textbf{How can I know the system is up and running?}

Docker comes with a set of tools which allows to inspect the state of a swarm.
With the following command, you can find out which cities are currently being
deployed:

\begin{lstlisting}[language=bash]
docker stack ls
\end{lstlisting}

Each city has a numerical identifier, starting from $0$: therefore, if you are
deploying two cities, you should see cities $0$ and $1$ after running the
command here above.

In order to know the status of a city, you have to issue the command (in which,
for example \texttt{<CITY\_ID>} is $0$):

\begin{lstlisting}[language=bash]
docker stack services <CITY_ID>
\end{lstlisting}

This command will cause Docker to output the list of services of a city: if
some of the services \emph{without} the \texttt{\_data} suffix does not have
any replica available for it, then you have to wait for it to be deployed.

Even if a service is started, it may not be ready: if you want to have more
information about it, you can inspect the logs of a single container with the
following command\footnote{where \texttt{<CONTAINER\_ID>} is the id of a Docker
container; you can find Docker container ids using tab completion in a shell}:

\begin{lstlisting}[language=bash]
docker logs -f <CONTAINER_ID>
\end{lstlisting}

As a rule of thumb, if no more logs are produced (when no simulation is
running), a container is ready; also, some of the services explicitly print
they are ready to run.
Then, you should be able to connect by means of a browser to the address given
in Section \ref{sec:sys-run}.
If you are not, then the server which provides you the Web pages may not be
ready: in that case, you can refresh the page from time to time until the
dashboard appears.

\end{itemize}
