\section{Frequently Asked Questions}

\begin{itemize}

\item \textbf{How can I use ACTS?}

ACTS provides three basic commands:
\begin{itemize}
	\item \texttt{build}
	\item \texttt{run}
	\item \texttt{clean}
\end{itemize}
You can learn how to use them by typing the specific command in your CLI or
by typing:
\begin{lstlisting}[language=bash]
<command_name> help
\end{lstlisting}
Each one will print to stdout an explanation of its usage.

More detailed instructions are provided in Section \ref{sec:sys}.

\item \textbf{How to set up a Docker Swarm?}

To set a local swam master for the simulation:

docker swarm init --advertise-addr <YOUR\_LOCAL\_IP\_ADDR>

More information can be found in the \href{https://docs.docker.com/engine/swarm/swarm-tutorial/create-swarm/}{docker swarm documentation}.
Also, you can check if the Docker Swarm started by following the instructions
of Section \ref{sec:sys-swarm}.

\item \textbf{How much time does Swarm takes to deploy the whole system?}

It depends on the number of cities and on the size of them\footnote{Clearly,
it is heavily affected by the computational resources of your machine}.
The estimated time with 2 cities, each one
composed by 3 districts, is around 5 minutes.
A small sized system, like
the one described, is composed by almost 30 containers.
However, it depends on the number or replicas declared in the configuration
files.
A replica is a duplicated container used by Docker Swarm to guarantee some
degree of availability of the service. This enables us to alleviate the single
point of failure problem at runtime. Indeed, if a container crashes, another
worker is ready to immediately replace it while Docker Swarm managers restart
the failing container.

\end{itemize}
