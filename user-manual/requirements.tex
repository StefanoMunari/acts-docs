\section{Requirements}\label{req}

The following dependencies are mandatory to use this product:
\begin{itemize}
   \item git - version 2.13.2 or greater;
   \item docker - version 17.06.0 or greater;
   \item docker-compose - version 1.17.0 or greater;
   \item chrome browser or firefox browser.
\end{itemize}

This project is heavily based on docker. Thus, the aforementioned
set of dependencies are light compared to all the other
technologies which compose the final product.

\subsection{Create a swarm}
After the dependencies have been successfully installed,
we are ready to create our local swarm\footnote{Due to the limited computational
resources, the project will be deployed on a single machine. However, this system
is effectively more suited to be deployed on a real distributed system}.

The swarm is necessary to easily deploy and manage the system which can be
viewed as composition of different units. To create a local swarm
type the following command:

docker swarm init --advertise-addr <YOUR_LOCAL_IP_ADDRESS>

Substitute <YOUR_LOCAL_IP_ADDRESS> with your local ip.
You can see the ip assigned to your machine by the local router with:

ip addr show | grep -A1 <NETWORK_INTERFACE>

We assume that you are running a *NIX OS. However the system can be installed
on any OS which fully supports the aforementioned dependencies.
Also, note that NETWORK_INTERFACE must
be replaced by the network interface associated with your IP (e.g., wlan0).


After the swarm has been successfully created, we need to instantiate a local
registry to store the project images. Do not worry if this registry figures as
an always running container, it is the standard case for it.
Before creating and running the registry, we have to verify that the port 5000
on our host is not already in use by another process.

You can check the port on your host with the following command:

sudo netstat -tulpn | grep LISTEN | grep 5000

If you do not have netstat on your system, another command is the following:

sudo lsof -i -P -n | grep LISTEN | grep 5000

If the result is empty (no entry returned) then the port 5000 is free.
We can safely run our registry:

docker service create --detach=false --name registry --publish 5000:5000 registry:latest

If you encounter any problem during these steps, then
consult the FAQ and troubleshooting sections to find (possibly)
helpful information.