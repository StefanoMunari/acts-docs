
\newcommand{\sv}{\mbox{Sebastiano} \mbox{Valle}\xspace}
\newcommand{\sm}{\mbox{Stefano} \mbox{Munari}\xspace}
\newcommand{\gm}{\mbox{Gianmarco} \mbox{Midena}\xspace}

%documenti
\newcommand{\RA}{\emph{Requirements Analysis}\xspace}

\newcommand{\productPurpose}{This product purpose is \dots}

\newcommand{\teacher}{Prof. \mbox{Vardanega} \mbox{Tullio}\xspace}
%\newcommand{\gruppo}{\xspace}
\newcommand{\project}{City Traffic\xspace}
\newcommand{\subject}{Concurrent and Distributed Systems\xspace}
%\newcommand{\groupmail}{\url{}\xspace}
\newcommand*{\customRef}[2]{\hyperref[{#1}]{#2 \ref*{#1}}}
\newcommand*{\sectionRef}[1]{\hyperref[{#1}]{§\ref*{#1}}}
\newcommand*{\hRef}[2]{\hyperref[{#1}]{#2}}

\newcommand{\nogloxy}[1]{#1} % comando da usare per evitare di metttere il mark del glossario
\newcommand{\gloxy}[1]{\emph{#1}$_G$}


\newcommand{\diaryEntry}[4]{#2 & \emph{#4} & #3 & #1\\ \hline}
\newcommand{\capitalizeFirstLetter}[1]{\StrLeft{#1}{1}[\temp]\uppercase\expandafter{\temp}\StrLen{#1}[\temp]\StrMid{#1}{2}{\temp}}
\newcommand{\conversationEntry}[2]{\textbf{\capitalizeFirstLetter{#1}}\\\\\emph{\indent \capitalizeFirstLetter{#2}}}
\newcommand{\chrule}{\begin{center}\line(1,0){250}\end{center}}

\newcommand{\si}{\rule{0pt}{4.8ex}\includegraphics[width=0.5cm]{template/icone/yes.pdf}\xspace}
\newcommand{\no}{\rule{0pt}{5.3ex}\includegraphics[width=0.5cm]{template/icone/no.pdf}\xspace}

\newcommand{\version}{\input{docInfo/version.tex}}

\newcommand{\descr}{Description}
\newcommand{\ops}{Operations}
\newcommand{\attrs}{Attributes}

%package names
\newcommand{\pApp}{app}
\newcommand{\pInterface}{\pApp::interface}
\newcommand{\pScheduling}{\pApp::scheduling}
\newcommand{\pReactive}{\pApp::reactive}
\newcommand{\pActive}{\pApp::active}
\newcommand{\pPassive}{\pApp::passive}
\newcommand{\pReactiveBuild}{\pReactive::build}
\newcommand{\pReactiveFactory}{\pReactive::factory}
\newcommand{\pReactiveComponent}{\pReactive::component}
\newcommand{\pReactiveComponentUtil}{\pReactiveComponent::util}
\newcommand{\pReactiveComponentStretch}{\pReactiveComponent::stretch}
\newcommand{\pReactiveComponentStretchDecoration}{\pReactiveComponentStretch::decoration}
\newcommand{\pReactiveComponentLane}{\pReactiveComponent::lane}
\newcommand{\pReactiveComponentLaneDecoration}{\pReactiveComponentLane::decoration}