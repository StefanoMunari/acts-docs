\documentclass[a4paper,11pt]{article}

\usepackage[british, UKenglish, USenglish, english, american]{babel}
\usepackage[utf8]{inputenc} % permette l'inserimento di caratteri accentati da tastiera nel documento sorgente.
\usepackage[T1]{fontenc} % specifica la codifica dei font da usare nel documento stampato.
\usepackage{lscape}
\usepackage{times} % per caricare un font scalabile
\usepackage{indentfirst} % rientra il primo capoverso di ogni unità di sezionamento.
\usepackage{titlesec}
\usepackage{makecell}
\usepackage{xspace}
\usepackage{xstring}
\usepackage{rotating, graphicx} % permette l'inserimento di immagini
\usepackage{multirow}
\usepackage{microtype} % migliora il riempimento delle righe
\usepackage{hyperref} % per gestione url
\hypersetup{
    colorlinks=true,       % false: boxed links; true: colored links
    linkcolor=black,          % color of internal links (change box color with linkbordercolor)
    citecolor=green,        % color of links to bibliography
    filecolor=magenta,      % color of file links
    urlcolor=blue           % color of external links
}
\usepackage{url} % per le url in monospace
\usepackage{eurosym} % simbolo euro
\usepackage{lastpage} % permette di sapere l'ultima pagina
\usepackage{fancyhdr} % gestione personalizzata header e footer
\usepackage[a4paper,portrait,top=3.5cm,bottom=3.5cm,left=3cm,right=3cm,bindingoffset=5mm]{geometry} % imposta i margini di pagina nelle classi standard.
\usepackage{hyperref} % abilita i riferimenti ipertestuali.
\usepackage{caption} %per le immagini
\usepackage{subcaption} %per le immagini
\usepackage{placeins} %per i floatbarrier
\usepackage{float} %per il posizionamento delle figure
\usepackage{verbatim} %per i commenti multiriga
\usepackage[table,usenames,dvipsnames]{xcolor}
\usepackage{longtable} % per le tabelle multipagina
\usepackage{diagbox}
\usepackage{hhline}
\usepackage{array} % per il testo nelle tabelle
\usepackage{multirow}
\usepackage{dirtree}
\usepackage{placeins} % \FloatBarrier per fare il flush delle immagini
\usepackage{tabularx} 
\usepackage{enumitem}
\usepackage{pifont}
\usepackage[normalem]{ulem}%testo sottolineato
\usepackage[bottom]{footmisc}

\usepackage[titletoc,title]{appendix}
\graphicspath{{./images/}} % da mettere per indicare le cartelle delle immagini

\let\stdsection\section
\renewcommand\section{\newpage\stdsection}

\lhead{\textsc\project}
\chead{}
\rhead{\leftmark}
\lfoot{\documenttitle}
\cfoot{}
\rfoot{Page: \thepage\ / \pageref{LastPage}}
\renewcommand{\headrulewidth}{0.4pt}
\renewcommand{\footrulewidth}{0.4pt}
\pagestyle{fancy}
\setlength{\headheight}{15pt}

\titleclass{\subsubsubsection}{straight}[\subsubsection]
\titleclass{\subsubsubsubsection}{straight}[\subsubsubsection]
\titleclass{\subsubsubsubsubsection}{straight}[\subsubsubsubsection]
\titleclass{\subsubsubsubsubsubsection}{straight}[\subsubsubsubsubsection]
\titleclass{\subsubsubsubsubsubsubsection}{straight}[\subsubsubsubsubsubsection]
\titleclass{\subsubsubsubsubsubsubsubsection}{straight}[\subsubsubsubsubsubsubsection]

\renewcommand\thesubsubsection{\thesubsection.\arabic{subsubsection}}
\newcounter{subsubsubsection}[subsubsection]
\renewcommand\thesubsubsubsection{\thesubsubsection.\arabic{subsubsubsection}}
\newcounter{subsubsubsubsection}[subsubsubsection]
\renewcommand\thesubsubsubsubsection{\thesubsubsubsection.\arabic{subsubsubsubsection}}
\newcounter{subsubsubsubsubsection}[subsubsubsubsection]
\renewcommand\thesubsubsubsubsubsection{\thesubsubsubsubsection.\arabic{subsubsubsubsubsection}}
\newcounter{subsubsubsubsubsubsection}[subsubsubsubsubsection]
\renewcommand\thesubsubsubsubsubsubsection{\thesubsubsubsubsubsection.\arabic{subsubsubsubsubsubsection}}
\newcounter{subsubsubsubsubsubsubsection}[subsubsubsubsubsubsection]
\renewcommand\thesubsubsubsubsubsubsubsection{\thesubsubsubsubsubsubsection.\arabic{subsubsubsubsubsubsubsection}}
\newcounter{subsubsubsubsubsubsubsubsection}[subsubsubsubsubsubsubsection]
\renewcommand\thesubsubsubsubsubsubsubsubsection{\thesubsubsubsubsubsubsubsection.\arabic{subsubsubsubsubsubsubsubsection}}

\titleformat{\subsubsubsection}
  {\normalfont\normalsize\bfseries}{\thesubsubsubsection}{1em}{}
\titlespacing*{\subsubsubsection}
{0pt}{3.25ex plus 1ex minus .2ex}{1.5ex plus .2ex}

\titleformat{\subsubsubsubsection}
  {\normalfont\normalsize\bfseries}{\thesubsubsubsubsection}{1em}{}
\titlespacing*{\subsubsubsubsection}
{0pt}{3.25ex plus 1ex minus .2ex}{1.5ex plus .2ex}

\titleformat{\subsubsubsubsubsection}
  {\normalfont\normalsize\bfseries}{\thesubsubsubsubsubsection}{1em}{}
\titlespacing*{\subsubsubsubsubsection}
{0pt}{3.25ex plus 1ex minus .2ex}{1.5ex plus .2ex}

\titleformat{\subsubsubsubsubsubsection}
  {\normalfont\normalsize\bfseries}{\thesubsubsubsubsubsubsection}{1em}{}
\titlespacing*{\subsubsubsubsubsubsection}
{0pt}{3.25ex plus 1ex minus .2ex}{1.5ex plus .2ex}

\titleformat{\subsubsubsubsubsubsubsection}
  {\normalfont\normalsize\bfseries}{\thesubsubsubsubsubsubsubsection}{1em}{}
\titlespacing*{\subsubsubsubsubsubsubsection}
{0pt}{3.25ex plus 1ex minus .2ex}{1.5ex plus .2ex}

\titleformat{\subsubsubsubsubsubsubsubsection}
  {\normalfont\normalsize\bfseries}{\thesubsubsubsubsubsubsubsubsection}{1em}{}
\titlespacing*{\subsubsubsubsubsubsubsubsection}
{0pt}{3.25ex plus 1ex minus .2ex}{1.5ex plus .2ex}

\makeatletter
\def\toclevel@subsubsection{3}
\def\toclevel@subsubsubsection{4}
\def\l@subsubsubsection{\@dottedtocline{4}{7em}{4em}}
\def\l@subsubsubsubsection{\@dottedtocline{5}{11em}{4em}}
\def\l@subsubsubsubsubsection{\@dottedtocline{6}{15em}{4em}}
\def\l@subsubsubsubsubsubsection{\@dottedtocline{7}{19em}{4em}}
\def\l@subsubsubsubsubsubsubsection{\@dottedtocline{8}{23em}{4em}}
\def\l@subsubsubsubsubsubsubsubsection{\@dottedtocline{9}{27em}{4em}}
\def\l@paragraph{\@dottedtocline{10}{10em}{5em}}
\def\l@subparagraph{\@dottedtocline{11}{14em}{6em}}
\makeatother

\setcounter{secnumdepth}{9}
\setcounter{tocdepth}{5}

\newcommand{\Cline}[2]{\noalign{\vskip-0.2pt}\hhline{|>{\arrayrulecolor{gray!30}}*{#1}{-}>{\arrayrulecolor{black}}|*{#2}{-}}\noalign{\vskip-0.2pt}}